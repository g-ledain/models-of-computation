\documentclass{article}
\usepackage[utf8]{inputenc}
\usepackage{amssymb}
\usepackage{mathtools}
\usepackage{hyperref}
\usepackage{parskip}
\usepackage{tikz}
\usepackage{amsmath}
\usepackage{amsthm}
\usepackage{comment}
\usepackage[none]{hyphenat}

\newtheoremstyle
  {break}% name
  {\topsep}% above space
  {\topsep}% below space
  {}% body font
  {}% indent
  {\bfseries}% head font
  {}% headpunct
  {\newline}%head space
  {}% custom head spec
\theoremstyle{break}

\newtheorem{theorem}{Theorem}[section]
\newtheorem{proposition}[theorem]{Proposition}
\newtheorem{notation}[theorem]{Notation}
\newtheorem{definition}[theorem]{Definition}
\newtheorem{lemma}[theorem]{Lemma}
\newtheorem{corollary}[theorem]{Corollary}
\newtheorem{conjecture}[theorem]{Conjecture}
\newtheorem{question}[theorem]{Question}
\newtheorem{example}[theorem]{Example}
\newtheorem{note}[theorem]{Note}
\newtheorem*{remark}{Remark}

\renewcommand\labelenumi{(\roman{enumi})}
\renewcommand\theenumi\labelenumi
\title{Models of computation}
\author{Gabriel Le Dain}
\date{March 2025}

\begin{document}
\title{Models of computation}
\maketitle

\section{Finite automata}

\subsection{Finite automata}
\begin{definition}[Finite automaton]
\label{def:finite-automaton}
A \textbf{finite automaton}, or \textbf{fintite state machine} (sometimes just \textbf{state machine}) is a $5$-tuple $(Q,\Sigma,\delta,q_0,F)$ where
\begin{enumerate}
    \item $Q$ is a finite set called the \textit{states}
    \item $\Sigma$ is a finite set called the \textit{alphabet}
    \item $\delta: Q\times\Sigma \to \Sigma$ is the \textit{transition function}
    \item $q_0\in Q$ is the \textit{start state}
    \item $F \subseteq Q$ is the set of \textit{final states} or \textit{accept states} 
\end{enumerate}
\end{definition}

\begin{notation}[Words over an alphabet]
\label{not:words}
Let $\Sigma$ be a set. The set of all words over $\Sigma$ is denoted by $\Sigma^*$.
This set of words always contains the empty string.
\end{notation}

\begin{notation}[Augmentation of an alphabet]
\label{not:augmentation}
For any alphabet $\Sigma$, we define the \textbf{augmentation} of $\Sigma$ as
\[\Sigma_{\varepsilon} := \Sigma \cup \{\varepsilon\}\]
where $\varepsilon$ is a formal symbol not in $\Sigma$.
We think of $\varepsilon$ as the empty string.
\end{notation}

\begin{definition}[Augmentation map]
\label{def:augmentation-map}
For any alphabet $\Sigma$, we define the augmentation map by
\[\textrm{aug}: \Sigma_{\varepsilon}* \to \Sigma^*\]
\[\textrm{aug}(\varepsilon) = \text{the empty string}\]
\[\textrm{aug}(a) = a \text{ for all } a\in \Sigma \]
extended to words over $\Sigma$ in the obvious way.
\end{definition}

\begin{definition}[Language]
A language over a set $\Sigma$ is some subset of $\Sigma^*$
\end{definition}

\begin{definition}[Accept]
\label{def:finite-automaton-accept}
Let $M=(Q,\Sigma,\delta,q_0,F)$ be a finite automation and $w = w_1w_2\ldots w_n$ be a word over $\Sigma$. 
We say that $M$ \textbf{accepts} $w$ if there exist $r_0, r_1, \ldots, r_n$ in $Q$ such that
\begin{enumerate}
  \item $r_0=q_0$
  \item $r_{i+1} = \delta(r_i, w_{i+1})$ for all $i = 0,1,\ldots, n-1$
  \item $r_n\in F$
\end{enumerate}
\end{definition}

\begin{definition}[Recognise]
\label{def:recognise}
Let $M=(Q,\Sigma,\delta,q_0,F)$ be a finite automaton.
We say that $M$ \textbf{recognises} the language 
\[L(M) := \{w \in \Sigma^*: M \text{ accepts } w\}\]
\end{definition}

\subsection{Regular languages}
\begin{definition}[Regular language]
\label{def:regular-language}
We say that a language $A$ over a set $\Sigma$ is a $\textbf{regular language}$ if there exists some finite automation $M$ with $\Sigma$ as its alphabet such that $M$ recognises $A$.
\end{definition}

\begin{definition}{Non-deterministic finite automaton}
\label{def:nondeterministic-finite-automaton}
A \textbf{non-deterministic finite automaton} is a $5$-tuple $(Q,\Sigma,\delta,q_0,F)$ where
\begin{enumerate}
    \item $Q$ is a finite set called the \textit{states}
    \item $\Sigma$ is a finite set called the \textit{alphabet}
    \item $\delta: Q\times\Sigma \to \mathcal{P}(Q)$ is the \textit{transition function}
    \item $q_0\in Q$ is the \textit{start state}
    \item $F \subseteq Q$ is the set of \textit{final states} or \textit{accept states} 
\end{enumerate}
\end{definition}

\begin{definition}[Accept]
\label{def:nondeterministic-finite-automaton-accept}
et $M=(Q,\Sigma,\delta,q_0,F)$ be a non-deterministic finite automaton and $w = w_1w_2\ldots w_n$ be a word over $\Sigma$. 
We say that $M$ \textbf{accepts} $w$ if there exist $r_0, r_1, \ldots, r_n$ in $Q$ such that
\begin{enumerate}
  \item $r_0=q_0$
  \item $r_{i+1} \in \delta(r_i, w_{i+1})$ for all $i = 0,1,\ldots, n-1$
  \item $r_n\in F$
\end{enumerate}
\end{definition}

\begin{definition}[$\varepsilon$-Non-deterministic finite automaton]
\label{def:epsilon-nondeterministic-finite-automaton}
  An \textbf{$\varepsilon$-non-deterministic finite automaton} is a $5$-tuple $(Q,\Sigma,\delta,q_0,F)$ where
  \begin{enumerate}
      \item $Q$ is a finite set called the \textit{states}
      \item $\Sigma$ is a finite set called the \textit{alphabet}
      \item $\delta: Q\times\Sigma_{\varepsilon} \to \mathcal{P}(Q)$ is the \textit{transition function}
      \item $q_0\in Q$ is the \textit{start state}
      \item $F \subseteq Q$ is the set of \textit{final states} or \textit{accept states} 
  \end{enumerate}
\end{definition}


\begin{definition}[Accept]
\label{def:epsilon-nondeterministic-finite-automaton-accept-aug}
Let $M:=(Q,\Sigma,\delta,q_0,F)$ be an $\varepsilon$-nondeterministic finite automaton.
We say that $M$ accepts a word $w \in \Sigma^*$ if there exists $w':=w_0w_1\ldots w_n \in\Sigma_{\varepsilon}^*$ and $r_0,r_1,\ldots,r_n \in Q$ such that $\textrm{aug}(w')=w$ and
  \begin{enumerate}
    \item $r_0=q_0$
    \item $r_{i+1} \in \delta(r_i, w_{i+1})$ for all $i = 0,1,\ldots, n-1$
    \item $r_n\in F$
  \end{enumerate}
\end{definition}

We can give a more compact definition of an $\varepsilon$-non-deterministic state machine accepting a word using the notion of the $\epsilon$-closure of a set.

\begin{definition}
  Let $Q,\Sigma$ be sets and 
  \[\delta: Q\times\Sigma_{\varepsilon} \to \mathcal{P}(Q) \]
  be a function. 
  Let $S \subseteq Q$. 
  Then the $\varepsilon$-closure of $S$ is the set
  \[\overline{S} := \{q \in Q: \exists q_1,\ldots,q_n \in S, q=q_1 \wedge \forall 1 \leq i < n, q_{i+1} \in \delta(q_i,\varepsilon) \}\]
\end{definition}


\begin{definition}[Accept]
  \label{def:epsilon-nondeterministic-finite-automaton-accept-aug-closure}
  Let $M:=(Q,\Sigma,\delta,q_0,F)$ be an $\varepsilon$-nondeterministic finite automaton.
  We say that $M$ accepts a word $w \in \Sigma^*$ if there exists $r_0,r_1,\ldots,r_n \in Q$ such that
    \begin{enumerate}
      \item $r_0=q_0$
      \item $r_{i+1} \in \overline{\delta(r_i, w_{i+1})}$ for all $i = 0,1,\ldots, n-1$
      \item $r_n\in F$
    \end{enumerate}
\end{definition}


\begin{definition}[Accept]
  \label{def:epsilon-nondeterministic-finite-automaton-accept-closure}
  Let $M:=(Q,\Sigma,\delta,q_0,F)$ be an $\varepsilon$-nondeterministic finite automaton.
  We say that $M$ accepts a word $w \in \Sigma^*$ if there exists $w:=w_1w_2\ldots w_n \in\Sigma^*$, $k_1, k_2, \ldots, k_n \in \mathbb{N}$ and $r_0,\ldots,r_n\in Q$ such that
    \begin{enumerate}
      \item $r_0=q_0, k_1=1$
      \item For all $i = 0,1,\ldots, n-1$, we have one of:
      \begin{enumerate}
        \item $r_{i+1} \in \delta(r_i, w_{k_{i+1}})$ and $k_{i+1}=k_i+1$
        \item $r_{i+1} \in \delta(r_i, \varepsilon)$ and $k_{i+1}=k_i$
      \end{enumerate}
      \item $r_n\in F$
    \end{enumerate}
\end{definition}

\begin{remark}
  The definitions \ref{def:epsilon-nondeterministic-finite-automaton-accept-aug} and \ref{def:epsilon-nondeterministic-finite-automaton-accept-closure} are equivalent by thinking about them for a second or two.
  However \ref{def:epsilon-nondeterministic-finite-automaton-accept-closure} is more suited to implementation in a programming language.
\end{remark}

There is an obvious way in which a finite automaton can be regarded as a non-deterministic finite automaton and in which a non-deterministic finite automaton can be regarded as a $\varepsilon$-non-deterministic finite automaton. 
Moreover, these interpretations leave unchanged the language that the automaton recognises.
More surprisingly, from a ($\varepsilon$)-non-deterministic automaton we can also construct a finite automaton which recognises the same language.
Thus, all three models of computation have the same expressive power.

\begin{definition}[Subset construction]
\label{def:subset-construction}
  Let $(Q,\Sigma,\delta,q_0,F)$ be a non-deterministic finite automaton. 
  We construct a finite automaton $(\mathcal{P}(Q), \Sigma, \delta', \{q_0\}, F')$ where
  \begin{itemize}
    \item $F' := \{S\in \mathcal{P}(Q): S\cap F \neq \emptyset\}$
    \item $\delta'(S, l) = \bigcup \{\delta(s,l): s\in S\}$
  \end{itemize}
\end{definition}

\begin{theorem}
\label{thm:subset-construction}
The subset construction leaves unchanged the language recognised by the automaton.
\end{theorem}

\begin{proof}(Of \ref{thm:subset-construction})
  Exercise.
\end{proof}

It would be nice if the equivalence $\textrm{finite automata} \leftrightarrow \varepsilon-\textrm{non-deterministic finite automata}$ factored as $\textrm{finite automata} \leftrightarrow \textrm{non-deterministic finite automata} \leftrightarrow \varepsilon-\textrm{non-deterministic finite automata}$.
This may well be doable (either on the present definitions or by modifying the definition of $\varepsilon$-non-deterministic finite automata to take the $\varepsilon$-closure \textit{before} the transition function) but we will go the simpler route of a direct construction $\varepsilon-\textrm{non-deterministic finite automata} \to \textrm{finite automata}$.

\begin{definition}[Subset construction]
\label{def:ep-subset-construction}
  Let $(Q,\Sigma,\delta,q_0,F)$ be an $\varepsilon$-non-deterministic finite automaton. 
  We construct a finite automaton $(\mathcal{P}(Q), \Sigma, \delta', \overline{\{q_0\}}, F')$ where
  \begin{itemize}
    \item $F' := \{S\in \mathcal{P}(Q): S\cap F \neq \emptyset\}$
    \item $\delta'(S, l) = \bigcup \{\overline{\delta(s,l)}: s\in S\}$
  \end{itemize}
\end{definition}

\begin{theorem}
\label{thm:ep-subset-construction}
The subset construction leaves unchanged the language recognised by the automaton.
\end{theorem}

\begin{proof}(Of \ref{thm:ep-subset-construction})
  Exercise.
\end{proof}


\subsection{Regular Languages}

\begin{definition}[Regular operations]
\label{def:regular-operation}
Let $\Sigma$ be some alphabet. 
We define three \textbf{regular operations} on languages:
\begin{enumerate}
  \item Union: $A \cup B$
  \item Concatenation: $A \circ B := \{ab: a\in A, b\in B\}$
  \item Kleene star: $A^* :=\{x_1x_2\ldots x_n \text{ where } x_1,x_2,\ldots,x_n\in A\}$
\end{enumerate}
\end{definition}

\begin{note}
\label{note:Kleene-star-compatible}
Note that the definition of the Kleene star on a language above is compatible with its use to denote the set of words over some alphabet.
\end{note}

We want to prove that applying the regular languages are closed under the regular operations.
In light of the equivalence of the three kinds of automata, it is enough to construct just an $\varepsilon$-non-deterministic finite automaton which recognises the union/concatenation/Kleene star of the languages.

For fun, let's consider intersection before any of the three regular operations:
\begin{proposition}[Intersections of regular languages are regular]
  \label{prop:intersection-of-regular-languages-regular}
    Let $A$, $B$ be regular languages over the same alphabet $\Sigma$. 
    Then $A\cap B$ is also regular.
  \end{proposition}
  
  \begin{proof}
  \label{prf:intersection-of-regular-languages-regular}
    Since $A$, $B$ are regular, they are recognised by finite automata $M^1:=(Q^1,\Sigma,\delta^1,q^1_0,F^1)$ and $M^2:=(Q^2,\Sigma,\delta^2,q^2_0,F^2)$.
    Define $M^3:=(Q^3,\Sigma,\delta^3,q^3_0,F^3)$ where:
    \begin{itemize}
      \item $Q^3 := Q^1 \times Q^2$
      \item $\delta^3((s^1, s^2), a) = (\delta^1(s^1,a). \delta^2(s^2, a))$
      \item $q^3_0 = (q^1_0, q^2_0)$
      \item $F^3 := F^1\times F^2$
    \end{itemize}
    Then $M_3$ recognises $A\cap B$.
  \end{proof}

For closure under union, we can actually construct a deterministic finite automaton.

\begin{proposition}[Unions of regular languages are regular]
\label{prop:union-of-regular-languages-regular}
Let $A$, $B$ be regular languages over the same alphabet $\Sigma$. 
Then $A\cup B$ is also regular.
\end{proposition}

\begin{proof}(Of \ref{prop:union-of-regular-languages-regular})
\label{prf:union-of-regular-languages-regular}
Since $A$, $B$ are regular, they are recognised by finite automata $M^1:=(Q^1,\Sigma,\delta^1,q^1_0,F^1)$ and $M^2:=(Q^2,\Sigma,\delta^2,q^2_0,F^2)$.
Define $M^3:=(Q^3,\Sigma,\delta^3,q^3_0,F^3)$ where:
\begin{itemize}
  \item $Q^3 := Q^1 \times Q^2$
  \item $\delta^3((s^1, s^2), a) = (\delta^1(s^1,a). \delta^2(s^2, a))$
  \item $q^3_0 = (q^1_0, q^2_0)$
  \item $F^3 := \left(F^1\times Q^2\right) \cup \left(Q^1\times F^2\right)$
\end{itemize}
Then $M_3$ recognises $A\cup B$.
\end{proof}

We can also construct an $\varepsilon$-non-determinstic automaton using a slightly different technique.

\begin{proof}(Of \ref{prop:union-of-regular-languages-regular})
\label{prf:union-of-regular-languages-regular-ep}
Since $A$, $B$ are regular, they are recognised by $\varepsilon$-non-deterministic finite automata $M^1:=(Q^1,\Sigma,\delta^1,q^1_0,F^1)$ and $M^2:=(Q^2,\Sigma,\delta^2,q^2_0,F^2)$.
Define $M^3:=(Q^3,\Sigma,\delta^3,q^3_0,F^3)$ where:
\begin{itemize}
  \item $Q^3 := Q^1 \coprod Q^2 \coprod \{\ast\}$
  \item $\delta^3(q,l) = \begin{cases}
  \{q^1_0, q^2_0\} & \text{if } q=\ast \text{ and } l = \varepsilon \\
  \delta^1(q,l) & \text{if } q\in Q^1 \\
  \delta^2(q,l) & \text{if } q\in Q^2 \\
  \emptyset & \text{otherwise} 
\end{cases}$
  \item $q^3_0 = \ast$
  \item $F^3 := F^1 \coprod F^2$
\end{itemize}
Then $M_3$ recognises $A\cup B$.
\end{proof}

The other two operations are harder to construct deterministic automata for, so we instead construct $\varepsilon$-non-deterministic automata for recognising them.

\end{document}
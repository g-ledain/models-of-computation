\documentclass{article}
\usepackage[utf8]{inputenc}
\usepackage{amssymb}
\usepackage{mathtools}
\usepackage{hyperref}
\usepackage{parskip}
\usepackage{tikz}
\usepackage{amsmath}
\usepackage{amsthm}
\usepackage{comment}
\usepackage[none]{hyphenat}

\newtheoremstyle
  {break}% name
  {\topsep}% above space
  {\topsep}% below space
  {}% body font
  {}% indent
  {\bfseries}% head font
  {}% headpunct
  {\newline}%head space
  {}% custom head spec
\theoremstyle{break}

\newtheorem{theorem}{Theorem}[section]
\newtheorem{proposition}[theorem]{Proposition}
\newtheorem{notation}[theorem]{Notation}
\newtheorem{definition}[theorem]{Definition}
\newtheorem{lemma}[theorem]{Lemma}
\newtheorem{corollary}[theorem]{Corollary}
\newtheorem{conjecture}[theorem]{Conjecture}
\newtheorem{question}[theorem]{Question}
\newtheorem{example}[theorem]{Example}
\newtheorem{note}[theorem]{Note}
\newtheorem*{remark}{Remark}

\renewcommand\labelenumi{(\roman{enumi})}
\renewcommand\theenumi\labelenumi
\title{Models of computation}
\author{Gabriel Le Dain}
\date{March 2025}

\begin{document}
\title{Models of computation}
\maketitle

\section{Finite automata}

\subsection{Finite automata}
\begin{definition}[Finite automaton]
\label{def:finite-automaton}
A \textbf{finite automaton}, or \textbf{fintite state machine} (sometimes just \textbf{state machine}) is a $5$-tuple $(Q,\Sigma,\delta,q_0,F)$ where
\begin{enumerate}
    \item $Q$ is a finite set called the \textit{states}
    \item $\Sigma$ is a finite set called the \textit{alphabet}
    \item $\delta: Q\times\Sigma \to \Sigma$ is the \textit{transition function}
    \item $q_0\in Q$ is the \textit{start state}
    \item $F \subseteq Q$ is the set of \textit{final states} or \textit{accept states} 
\end{enumerate}
\end{definition}

\begin{notation}[Words over an alphabet]
\label{not:words}
Let $\Sigma$ be a set. The set of all words over $\Sigma$ is denoted by $\Sigma^*$.
This set of words always contains the empty string.
\end{notation}

\begin{notation}[Augmentation of an alphabet]
\label{not:augmentation}
For any alphabet $\Sigma$, we define the \textbf{augmentation} of $\Sigma$ as
\[\Sigma_{\varepsilon} := \Sigma \cup \{\varepsilon\}\]
where $\varepsilon$ is a formal symbol not in $\Sigma$.
We think of $\varepsilon$ as the empty string.
\end{notation}

\begin{definition}[Augmentation map]
\label{def:augmentation-map}
For any alphabet $\Sigma$, we define the augmentation map by
\[\textrm{aug}: \Sigma_{\varepsilon}* \to \Sigma^*\]
\[\textrm{aug}(\varepsilon) = \text{the empty string}\]
\[\textrm{aug}(a) = a \text{ for all } a\in \Sigma \]
extended to words over $\Sigma$ in the obvious way.
\end{definition}

\begin{definition}[Language]
A language over a set $\Sigma$ is some subset of $\Sigma^*$
\end{definition}

\begin{definition}[Accept]
\label{def:finite-automaton-accept}
Let $M=(Q,\Sigma,\delta,q_0,F)$ be a finite automation and $w = w_1w_2\ldots w_n$ be a word over $\Sigma$. 
We say that $M$ \textbf{accepts} $w$ if there exist $r_0, r_1, \ldots, r_n$ in $Q$ such that
\begin{enumerate}
  \item $r_0=q_0$
  \item $r_{i+1} = \delta(r_i, w_{i+1})$ for all $i = 0,1,\ldots, n-1$
  \item $r_n\in F$
\end{enumerate}
\end{definition}

\begin{definition}[Recognise]
\label{def:recognise}
Let $M=(Q,\Sigma,\delta,q_0,F)$ be a finite automaton.
We say that $M$ \textbf{recognises} the language 
\[L(M) := \{w \in \Sigma^*: M \text{ accepts } w\}\]
\end{definition}

\subsection{Regular languages}
\begin{definition}[Regular language]
\label{def:regular-language}
We say that a language $A$ over a set $\Sigma$ is a $\textbf{regular language}$ if there exists some finite automation $M$ with $\Sigma$ as its alphabet such that $M$ recognises $A$.
\end{definition}

\begin{definition}{Non-deterministic finite automaton}
\label{def:nondeterministic-finite-automaton}
A \textbf{non-deterministic finite automaton} is a $5$-tuple $(Q,\Sigma,\delta,q_0,F)$ where
\begin{enumerate}
    \item $Q$ is a finite set called the \textit{states}
    \item $\Sigma$ is a finite set called the \textit{alphabet}
    \item $\delta: Q\times\Sigma \to \mathcal{P}(Q)$ is the \textit{transition function}
    \item $q_0\in Q$ is the \textit{start state}
    \item $F \subseteq Q$ is the set of \textit{final states} or \textit{accept states} 
\end{enumerate}
\end{definition}

\begin{definition}[Accept]
\label{def:nondeterministic-finite-automaton-accept}
et $M=(Q,\Sigma,\delta,q_0,F)$ be a non-deterministic finite automaton and $w = w_1w_2\ldots w_n$ be a word over $\Sigma$. 
We say that $M$ \textbf{accepts} $w$ if there exist $r_0, r_1, \ldots, r_n$ in $Q$ such that
\begin{enumerate}
  \item $r_0=q_0$
  \item $r_{i+1} \in \delta(r_i, w_{i+1})$ for all $i = 0,1,\ldots, n-1$
  \item $r_n\in F$
\end{enumerate}
\end{definition}

\begin{definition}[$\varepsilon$-Non-deterministic finite automaton]
\label{def:epsilon-nondeterministic-finite-automaton}
  An \textbf{$\varepsilon$-non-deterministic finite automaton} is a $5$-tuple $(Q,\Sigma,\delta,q_0,F)$ where
  \begin{enumerate}
      \item $Q$ is a finite set called the \textit{states}
      \item $\Sigma$ is a finite set called the \textit{alphabet}
      \item $\delta: Q\times\Sigma_{\varepsilon} \to \mathcal{P}(Q)$ is the \textit{transition function}
      \item $q_0\in Q$ is the \textit{start state}
      \item $F \subseteq Q$ is the set of \textit{final states} or \textit{accept states} 
  \end{enumerate}
\end{definition}


\begin{definition}[Accept]
\label{def:epsilon-nondeterministic-finite-automaton-accept-aug}
Let $M:=(Q,\Sigma,\delta,q_0,F)$ be an $\varepsilon$-nondeterministic finite automaton.
We say that $M$ accepts a word $w \in \Sigma^*$ if there exists $w':=w_0w_1\ldots w_n \in\Sigma_{\varepsilon}^*$ and $r_0,r_1,\ldots,r_n \in Q$ such that $\textrm{aug}(w')=w$ and
  \begin{enumerate}
    \item $r_0=q_0$
    \item $r_{i+1} \in \delta(r_i, w_{i+1})$ for all $i = 0,1,\ldots, n-1$
    \item $r_n\in F$
  \end{enumerate}
\end{definition}

\begin{definition}[Accept]
  \label{def:epsilon-nondeterministic-finite-automaton-accept}
  Let $M:=(Q,\Sigma,\delta,q_0,F)$ be an $\varepsilon$-nondeterministic finite automaton.
  We say that $M$ accepts a word $w \in \Sigma^*$ if there exists $w:=w_1w_2\ldots w_n \in\Sigma^*$, $k_1, k_2, \ldots, k_n \in \mathbb{N}$ and $r_0,\ldots,r_n\in Q$ such that
    \begin{enumerate}
      \item $r_0=q_0, k_1=1$
      \item For all $i = 0,1,\ldots, n-1$, we have one of:
      \begin{enumerate}
        \item $r_{i+1} \in \delta(r_i, w_{k_{i+1}})$ and $k_{i+1}=k_i+1$
        \item $r_{i+1} \in \delta(r_i, \varepsilon)$ and $k_{i+1}=k_i$
      \end{enumerate}
      \item $r_n\in F$
    \end{enumerate}
\end{definition}

\begin{remark}
  The definitions \ref{def:epsilon-nondeterministic-finite-automaton-accept} and \ref{def:epsilon-nondeterministic-finite-automaton-accept-aug} are equivalent by thinking about them for a second or two.
  However \ref{def:epsilon-nondeterministic-finite-automaton-accept} is more suited to implementation in a programming language.
\end{remark}

\begin{definition}[Regular operations]
\label{def:regular-operation}
Let $\Sigma$ be some alphabet. 
We define three \textbf{regular operations} on languages:
\begin{enumerate}
  \item Union: $A \cup B$
  \item Concatenation: $A \circ B := \{ab: a\in A, b\in B\}$
  \item Kleene star: $A^* :=\{x_ax_2\ldots x_n \text{ where } x_1,x_2,\ldots,x_n\in A\}$
\end{enumerate}
\end{definition}

\begin{note}
\label{note:Kleene-star-compatible}
Note that the definition of the Kleene star on a language above is compatible with its use to denote the set of words over some alphabet.
\end{note}

\begin{proposition}[Unions of regular languages are regular]
\label{prop:union-of-regular-languages-regular}
Let $A$, $B$ be regular languages over the same alphabet $\Sigma$. 
Then $A\cup B$ is also regular.
\end{proposition}

\begin{proof}
\label{prf:union-of-regular-languages-regular}
Since $A$, $B$ are regular, they are recognised by finite automata $M^1:=(Q^1,\Sigma,\delta^1,q^1_0,F^1)$ and $M^2:=(Q^2,\Sigma,\delta^2,q^2_0,F^2)$.
Define $M^3:=(Q^3,\Sigma,\delta^3,q^3_0,F^3)$ where:
\begin{itemize}
  \item $Q^3 := Q^1 \times Q^2$
  \item $\delta^3((s^1, s^2), a) = (\delta^1(s^1,a). \delta^2(s^2, a))$
  \item $q^3_0 = (q^1_0, q^2_0)$
  \item $F^3 := \left(F^1\times Q^2\right) \cup \left(Q^1\times F^2\right)$
\end{itemize}
Then $M_3$ recognises $A\cup B$.
\end{proof}

\begin{proposition}[Intersections of regular languages are regular]
\label{prop:intersection-of-regular-languages-regular}
  Let $A$, $B$ be regular languages over the same alphabet $\Sigma$. 
  Then $A\cap B$ is also regular.
\end{proposition}

\begin{proof}
\label{prf:intersection-of-regular-languages-regular}
  Since $A$, $B$ are regular, they are recognised by finite automata $M^1:=(Q^1,\Sigma,\delta^1,q^1_0,F^1)$ and $M^2:=(Q^2,\Sigma,\delta^2,q^2_0,F^2)$.
  Define $M^3:=(Q^3,\Sigma,\delta^3,q^3_0,F^3)$ where:
  \begin{itemize}
    \item $Q^3 := Q^1 \times Q^2$
    \item $\delta^3((s^1, s^2), a) = (\delta^1(s^1,a). \delta^2(s^2, a))$
    \item $q^3_0 = (q^1_0, q^2_0)$
    \item $F^3 := F^1\times F^2$
  \end{itemize}
  Then $M_3$ recognises $A\cap B$.
\end{proof}

\end{document}